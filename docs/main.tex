\documentclass[a4paper,article]{memoir}

\usepackage[utf8]{inputenc}
\usepackage[UKenglish]{babel}
\usepackage[T1]{fontenc}
\usepackage[margin=3cm]{geometry}
\usepackage{graphicx}
\usepackage{xspace}
\usepackage{xcolor}
\usepackage{array,booktabs}
\usepackage{url}
\usepackage[autostyle,danish=guillemets]{csquotes}
\usepackage{lipsum}
\usepackage{cite}


\usepackage{amsmath,amssymb}
\usepackage{bm}
\usepackage{amsthm}
\usepackage{mathtools}
\usepackage{listings}
\usepackage{verbatim}

\usepackage{physics}
\usepackage{siunitx}
\usepackage{isotope}


\usepackage[draft,danish]{fixme}


\usepackage{hyperref}
\hypersetup{
	colorlinks,
	linkcolor={red!50!black},
	citecolor={blue!50!black},
	urlcolor={blue!80!black}
}

\newcommand{\lag}{\mathcal{L}}

\newcommand{\psibar}{\overline{\psi}}

\newcommand{\fermmatsym}{M}
\newcommand{\fermmat}{\fermmatsym\left[U\right]}

\newcommand{\gaugeint}{\int \mathcal{D}U}
\newcommand{\quarkint}{\int \mathcal{D} \left( \psi,\psibar \right)}

\newcommand{\Dw}{D_{\mathrm{W}}}

\newcommand{\SU}{SU(3)}
\newcommand{\Nf}{N_\mathrm{f}}

\newcommand{\tens}[2]{\tensor{#1}{#2}}

\newcommand{\csw}{c_{\mathrm{SW}}}

\newcommand{\re}{\mathrm{Re}}

\newcommand{\expp}[1]{e^{#1}}

\newcommand{\Nev}{N_{\mathrm{ev}}}

\newcommand{\Ecm}{E_{\mathrm{cm}}}
\newcommand{\qcmv}{\vb{q}_{\mathrm{cm}}}
\newcommand{\qcm}{q_{\mathrm{cm}}}

\newcommand{\Ktilde}{\tilde{K}}
\newcommand{\Lmax}{L_{\mathrm{max}}}

\newcommand{\Nboot}{N_{\mathrm{boot}}}

\newcommand{\ndof}{\mathrm{d.o.f}}
\newcommand{\tmin}{t_{\mathrm{min}}}



% ====================================================================
\geometry{headheight=1cm}
\title{MMCT Mathematical background}
\newcommand{\subtitle}{Subtitle} 
%\renewcommand{\maketitlehookb}{\centering\textsc{\subtitle}}
\author{Chris Walther Andersen}
%\date{}
% ====================================================================



\begin{document}
	
\setlength{\parindent}{0pt}

\maketitle
\fancybreak{$*\quad*\quad*$}
\vspace{5mm}

\chapter{Introduction}
This document describes the mathematical background of the MMCT pyhton package, 
which can be used to do tests of multinomial data with monte carlo sampling. 
This document does not attempt to explain how to use the code. Instead look at 
the GitHub page for inspiration.

\chapter{The multinomial distribution}
The multinomial distribution is the natural extension of the binomial 
distribution, in which a random variable can fall in one of two cases with 
probability $p$ and $1-p$. In the multinomial distribution there are $k$ 
different possibilities (bins) for the random variable, and accordingly each of 
the 
$k$ bins are reached with probabilities $p_1, p_2, \ldots p_k$, such 
that
\begin{equation}
	\sum_{i=1}^k p_i = 1.
\end{equation}
We imagine drawing $n$ times from probability distribution. We end up with 
$x_1$ draws in bin 1, $x_2$ draws in bin 2 and so on:
\begin{equation}
	\sum_{i=1}^k x_i = n.
\end{equation}
The probability density function (the probability of drawing exactly $x_1, x_2, 
\ldots, x_k$ given $p_1, p_2, \ldots, p_k$) is
\begin{equation}
	f(x_1,\ldots, x_k;n,p_1,\ldots,p_k) = f(x_i;n,p_i) = \frac{n!}{x_1! \cdots 
	x_k!} p_1^{x_1} \cdots p_k^{x_k}.
\end{equation}
This probability will almost always be very small. For example, rolling a single
D6 three times and getting 3, 4 and 5 eyes (in any order) is not unreasonable, 
however the probability is only a bit above 3\%. To test whether the result is
unreasonable we need to know if the observed outcome is \emph{more} unlikely 
than other outcomes. For example rolling the D6 three times and getting 3 eyes 
each time has a probability of just 0.4\%.

 

 
\bibliographystyle{JHEP}
\bibliography{bub}
	
	
\end{document}
