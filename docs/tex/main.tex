%% Overleaf			
%% Software Manual and Technical Document Template	
%% 									
%% This provides an example of a software manual created in Overleaf.

\documentclass{ol-softwaremanual}

% Packages used in this example
\usepackage{graphicx}  % for including images
\usepackage{microtype} % for typographical enhancements
\usepackage{minted}    % for code listings
\usepackage{amsmath}   % for equations and mathematics
\setminted{style=friendly,fontsize=\small}
\renewcommand{\listoflistingscaption}{List of Code Listings}
\usepackage{hyperref}  % for hyperlinks
\usepackage[a4paper,top=4.2cm,bottom=4.2cm,left=3.5cm,right=3.5cm]{geometry} % 
%for setting page size and margins
\usepackage{csquotes}

% Custom macros used in this example document
\newcommand{\doclink}[2]{\href{#1}{#2}\footnote{\url{#1}}}
\newcommand{\cs}[1]{\texttt{\textbackslash #1}}
\newcommand{\mmct}{\texttt{mmct}}

% Frontmatter data; appears on title page
\title{MMCT Python Package \\ Documentation}
\version{1.1.2}
\author{Chris Walther Andersen}
\softwarelogo{\includegraphics[width=8cm]{logo}}

\begin{document}

\maketitle

\tableofcontents
\listoflistings
\newpage

\section{Introduction}

The purpose of this software is to perform a statistical test on a dataset, in 
order to determine whether the data has been generated from a multinomial 
distribution. The name of the package, \mmct, is an abbreviation of 
\enquote{Multinomial Monte Carlo Test}. As the name suggests, the statistical 
test is performed using a Monte Carlo simulation. This document describes how 
this is done and how to interact with the code to perform a test.

%
%
%Once you're familiar with the editor, you can find various project settings in 
%the Overleaf menu, accessed via the button in the very top left of the editor. 
%To view tutorials, user guides, and further documentation, please visit our 
%\doclink{https://www.overleaf.com/learn}{help library}. If you haven't used 
%\LaTeX\ before, our 
%\doclink{https://www.overleaf.com/learn/latex/Learn_LaTeX_in_30_minutes}{Learn 
%\LaTeX\ in 30 minutes} tutorial is an excellent place to start.


\section{Including code samples}

\input{code-samples}


\end{document}
