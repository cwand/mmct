%% Overleaf			
%% Software Manual and Technical Document Template	
%% 									
%% This provides an example of a software manual created in Overleaf.

\documentclass{ol-softwaremanual}

% Packages used in this example
\usepackage{graphicx}  % for including images
\usepackage{microtype} % for typographical enhancements
\usepackage{minted}    % for code listings
\usepackage{amsmath}   % for equations and mathematics
\setminted{style=friendly,fontsize=\small}
\renewcommand{\listoflistingscaption}{List of Code Listings}
\usepackage{hyperref}  % for hyperlinks
\usepackage[a4paper,top=4.2cm,bottom=4.2cm,left=3.5cm,right=3.5cm]{geometry} % 
%for setting page size and margins
\usepackage{csquotes}
\usepackage{nth}

% Custom macros used in this example document
\newcommand{\doclink}[2]{\href{#1}{#2}\footnote{\url{#1}}}
\newcommand{\cs}[1]{\texttt{\textbackslash #1}}
\newcommand{\mmct}{\texttt{mmct}}

% Frontmatter data; appears on title page
\title{MMCT Python Package \\ Documentation}
\version{1.1.2}
\author{Chris Walther Andersen}
\softwarelogo{\includegraphics[width=8cm]{logo}}

\begin{document}

\maketitle

\tableofcontents
\listoflistings
\newpage

\section{Introduction}

The purpose of this software is to perform a statistical test on a dataset, in 
order to determine whether the data has been generated from a multinomial 
distribution. The name of the package, \mmct, is an abbreviation of 
\enquote{Multinomial Monte Carlo Test}. As the name suggests, the statistical 
test is performed using a Monte Carlo simulation. This document describes how 
this is done and how to interact with the code to perform a test.

The package is inspired by two other software packages, wich boh perform the 
same job:
The Python-package \doclink{https://pypi.org/project/met/}{met} is also a 
Python-package for performing multinomial tests, however \texttt{met} performs 
an \emph{exact} test. While this strategy is certainly preferred to the Monte 
Carlo version (which will always be an approximate test), an exact test quickly 
becomes infeasible as the problem size grows. The other package to have 
inspired this project is the 
\doclink{https://cran.r-project.org/web/packages/XNomial/vignettes/XNomial.html}{XNomial}
package for the R-programming language. \mmct is basically an XNomial-clone for 
Pyhton.



\section{Mathematical background}

Assume we have done an experiment in which $N$ observations of a system that 
can end up in one of $k$ different states has been made (e.g. rolling a D6 20 
times: $N=20$, $k=6$). We have tracked the number of times each state occurred 
and denote them $m_1, m_2, \ldots, m_k$, such that $\sum_{i=1}^k m_i = N$.

We want to test the hypothesis that this result is compatible with a 
multinomial distribution with parameters $N$ and $p_1, p_2, \ldots, p_k$, where 
$p_i$ is the probability of the state $i$ occurring, where $\sum_{i=1}^k p_i = 
1$. The result of this test should be a p-value for the null hypothesis that 
$m_1, \ldots, m_k$ are drawn from this multinomial distribution.

\subsection{Monte Carlo sampling}

The test is performed in two steps: Monte Carlo sampling a bunch of 
multinomially distributed sets of data, and afterwards determining the 
likelihood of the original dataset occuring based on the sampled data.

First we write the original data under test as
\begin{equation}
X^0 = \left\{ m^0_1, m^0_2, \ldots , m^0_k \right\},
\end{equation}
where $m^0_i$ is the number of observations in bin $i$ and again 
$m^0_1+m^0_2+\ldots+m^0_k = N$.

We start the test by sampling from the hypothesised multinomial distribution 
$N$ times. 
We will end up with a number of observations in each of the $k$ bins, each 
drawn with probability $p_i$, so we end up with
\begin{equation}
X^1 = \left\{ m^1_1, m^1_2, \ldots , m^1_k \right\},
\end{equation}
such that $m^1_1 + m^1_2 + \ldots + m^1_k = N$. We save this result, and then 
start over by sampling another $N$ times into a new result:
\begin{equation}
X^2 = \left\{ m^2_1, m^2_2, \ldots , m^2_k \right\}.
\end{equation}
We continue this process $N_S$ times, ending up with $N_S$ random result 
vectors $X^j, 1\leq j \leq N_S$, each generated from the same, hypothesised 
multinomial distribution. If 
$N_S$ is large enough, we can now estimate the likelihood that $X^0$ (the data 
under test) is drawn 
from the same distribution by comparing it to the randomly generated samples.

\subsection{Sample grading}

To do this we need a way of grading our samples to determine what is likely and 
what is unlikely. The \mmct package implements two different ways of doing this:

\subsubsection{Multinomial probability function}

The first and most obvious way would be to grade each sample by its raw 
probability of occurring according to the probability distribution. This 
probability is calculates as
\begin{align}
P^j &= \Pr\left(m^j_1, \ldots, m^j_k | N , p_1, \ldots, p_k\right) \nonumber \\ 
\label{eq:prob}
&= \frac{N!}{m^j_1! \cdots m^j_k!} p_1^{m^j_1} \cdots p_k^{m^j_k}.
\end{align}
Calculating $P$ for all samples and our data under test in this way, we can 
now rank all the samples from the most likely (largest $P$) to most unlikely 
(smallest $P$) and look where $P^0$, the rank of the data under test, fits in. 
The p-value of our test is simply the number of samples with a probability 
$P^j$ smaller than $P^0$ divided by the total number of samples $N_S$:
\begin{equation}
\textrm{p-value} = \frac{N_{\textrm{smaller}}}{N_S}
\end{equation}


\subsubsection{Likelihood ratio}

The second way uses likelihood-ratios. For any given Monte Carlo sample $X^j$ 
we can calculate the likelihood-function $\mathcal{L}$ evaluated as the 
likelihood of the hypothesised probabilities given the sample $X^j$, which is 
simply equation \eqref{eq:prob}. In the spirit of the 
\doclink{https://en.wikipedia.org/wiki/Likelihood-ratio_test}{likelihood-ratio 
test} we can compare this to an alternate hypothesis. The optimal alternate 
hypothesis is just to use the probabilities observed in the original data 
$X^0$, i.e.
\begin{align*}
\hat{p}_1 &= \frac{m^0_1}{N} \\
\hat{p}_2 &= \frac{m^0_2}{N} \\
& \cdots \\
\hat{p}_k &= \frac{m^0_k}{N}.
\end{align*}
The likelihood function of these probabilities, given the Monte Carlo sample 
$X^j$, is
\begin{equation}
\frac{N!}{m^j_1! \cdots m^j_k!} \hat{p}_1^{m^j_1} \cdots \hat{p}_k^{m^j_k}
\end{equation}


\section{Including code samples}

\input{code-samples}


\end{document}
